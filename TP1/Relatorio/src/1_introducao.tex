\chapter[Introdução]{Introdução}
\label{cap:introducao}

O problema da mochila é um problema de optimização combinatória. A situação pode ser descrita como preencher uma mochila com o maior valor possível uma vez que ela somente suporta um determinado peso e cada objeto tem um valor e peso bem definido.

Para este trabalho usamos as estratégias de algoritmos guloso e tentativa e erro. A aproximação pelo algoritmo guloso consiste em primeiro ordenar o vetor de alguma forma, seja por peso ou por valor e então tomar decisões baseadas nas informações com base elas são adquiridas, sem voltar atrás. Ou seja, uma vez que um objeto foi colocado na mochila, ele não poderá ser retirado. A aproximação por tentativa e erro, ou backtracking pode ser definida como criar uma árvore de possibilidades e então percore-la até que se ache a "saida" ou alguma solução viável. 

Ambos os algoritmos não apresentam soluções ótimas, porém são estrategias válidas.
