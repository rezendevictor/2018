%% Baseado no arquivo: 
%% abtex2-modelo-trabalho-academico.tex, v-1.9.6 laurocesar
%% by abnTeX2 group at http://www.abntex.net.br/ 
%% Adaptado para um modelo de TCC (Graduação)

% ---
% Capa
% ---
\imprimircapa
% ---

% ---
% Folha de rosto
% (o * indica que haverá a ficha bibliográfica)
% ---
\imprimirfolhaderosto*
% ---



% \vspace{\onelineskip}

% FERRIGNO, C. R. A. \textbf{Tratamento de neoplasias ósseas apendiculares com
% reimplantação de enxerto ósseo autólogo autoclavado associado ao plasma
% rico em plaquetas}: estudo crítico na cirurgia de preservação de membro em
% cães. 2011. 128 f. Tese (Livre-Docência) - Faculdade de Medicina Veterinária e
% Zootecnia, Universidade de São Paulo, São Paulo, 2011.

% \begin{table}[htb]
% \center
% \footnotesize
% \begin{tabular}{|p{1.4cm}|p{1cm}|p{3cm}|p{3cm}|}
%   \hline
%    \textbf{Folha} & \textbf{Linha}  & \textbf{Onde se lê}  & \textbf{Leia-se}  \\
%     \hline
%     1 & 10 & auto-conclavo & autoconclavo\\
%    \hline
% \end{tabular}
% \end{table}

% \end{errata}
% ---



% ---
% RESUMO
% ---

% resumo em português
\setlength{\absparsep}{18pt} % ajusta o espaçamento dos parágrafos do resumo
\begin{resumo}
Este trabalho busca resolver o "Problema da Mochila", utilizando dos metódos de "tentativa e erro" e "aloritmos guloso". Então será apresentado uma análise sobre os algortimos apresentados. O problema é um problema clássico para programação dinâmica e desenvolvimento de software.
Logo após, apresente os resultados obtidos, contribuições e conclusões obtidas.

 \textbf{Palavras-chave}: Algoritmos. Programação Dinâmica. Mochila.
\end{resumo}



% ---
% inserir o sumario
% ---
\pdfbookmark[0]{\contentsname}{toc}
\tableofcontents*
\cleardoublepage
% ---